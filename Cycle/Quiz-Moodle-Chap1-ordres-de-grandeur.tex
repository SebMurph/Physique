\documentclass[12pt]{article}
\usepackage{moodle}
\moodleregisternewcommands
\newcommand\monomial[1]{x^{#1}}
\newcommand\sillyanswer{What!?}
\begin{document}
\begin{quiz}{Ordres de grandeur}


\begin{multi}[points=1]{Distance Terre Lune}
La distance Terre-Lune est $d$ = 384\,400 km. Quel est l’ordre de grandeur de cette distance en km ?
\item [feedback= mauvaise réponse écrire ce nombre en écriture scientifique et regarder la puissance de dix correspondante] $10^4$ km $10^3$ km
\item[feedback={yes!}] * $10^5$ km
\item[feedback= mauvaise réponse écrire ce nombre en écriture scientifique et regarder la puissance de dix correspondante] $10^6$ km
\item[feedback= mauvaise réponse écrire ce nombre en écriture scientifique et regarder la puissance de dix correspondante] $10^4$ km
\end{multi}

\begin{multi}[points=1]{Vitesse de la lumière}
La vitesse de la lumière dans le vide est de 300\,000 km/s. Quel est l’ordre de grandeur de cette vitesse?
\item $10^4$ km/s
\item[feedback={yes!}] * $10^5$ km/s
\item[feedback= {mauvaise réponse, ceci représente une écriture scientifique du nombre, pour l'ordre de grandeur il faut prendre uniquement la puissance de dix}] $3 \cdot 10^5$ km/s
\item[feedback= mauvaise réponse] $10^6$ km/s
\end{multi}

\begin{multi}[points=1]{Atome d'hydrogène}
Le diamètre d'un atome d'hydrogène est de 106 pm. Quel est l'ordre de grandeur de ce diamètre en m ?
\item $10^{-2}$ m
\item  * $10^{-10}$ m
\item[feedback= {mauvaise réponse, écrire ce nombre en écriture scientifique et regarder la puissance de dix correspondante}] $10^{-12}$ m
\item[feedback= mauvaise réponse écrire ce nombre en écriture scientifique et regarder la puissance de dix correspondante] $10^{-14}$ m
\item[feedback= mauvaise réponse écrire ce nombre en écriture scientifique et regarder la puissance de dix correspondante] $10^{-8}$ m
\end{multi}

\begin{multi}[points=1]{goutte d'eau}
Le diamètre d'une goutte d'eau est de 2 mm environ. Quel est l'ordre de grandeur du diamètre d'une goutte d'eau?
\item 1 m
\item   * 1 mm
\item $10^{-5}$ m
\item $10^{-2}$ mm
\item 1 cm
\end{multi}

\begin{multi}[points=1]{grain de sable}
À ton avis quel est l'ordre de grandeur d'un grain de sable ? 
\item 1 m
\item   * 1 mm
\item 1 $\mu$m
\item 1 nm
\item 1 cm
\end{multi}

\begin{multi}[points=1]{salle de cours}
À ton avis quel est l'ordre de grandeur de la longueur de la salle de cours de physique ?
\item 1 m
\item   * 10 m
\item 100 m
\item 1 hm
\item 1 km
\end{multi}

\begin{multi}[points=1]{Saturne}
Saturne est à environ 1,3 milliard de km de la Terre. Quel est l'ordre de grandeur en km de la distance Terre Saturne?.
\item 10 milliard de km
\item   * 1 milliard de km
\item 2 milliards de km
\item 100 millions de km
\item 10 millions de km
\end{multi}

\begin{multi}[points=1]{Distance Terre-Soleil}
Le Soleil est à $1,5\cdot10^8$ km de la Terre. Quel est l'ordre de grandeur, en km, de la distance Terre-Soleil?.
\item 10 milliard de km
\item   * 100 millions de km
\item 100 milliards de km
\item 100 mille de km
\item 10 millions de km
\end{multi}

\begin{multi}[points=1]{Nanomètre}
Dans la liste ci-dessous, qu'est-ce qui pourrait avoir une taille de 1 nm ?
\item une cellule [feedback={Une cellule a un ordre de grandeur du micromètre et non du nanomètre}]
\item  * une molécule
\item une poussière
\item un grain de sable
\item le noyau d'un atome [feedback={le noyau des atomes sont de l'ordre de grandeur du femtomètre et non du nanomètre}]
\end{multi}

\begin{multi}[points=1]{Milimètre}
Dans la liste ci-dessous, qu'est-ce qui pourrait avoir une taille de 2 mm ?
\item une fourmi
\item   * un grain de sable 
\item une poussière
\item une molécule
\item un atome
\end{multi}

\begin{multi}[points=1]{Foron}
Une salle de cours au CO Foron fait 10 mètres environ et le cycle mesure à peu près 220 mètres. En ordre de grandeur on peut dire que...
\item le CO Foron est environ cent fois plus long qu'une salle de classe
\item * le CO Foron est environ dix fois plus long qu'une salle de classe
\item le CO Foron est environ mille fois plus long qu'une salle de classe
\item le CO Foron est environ la même longueur qu'une salle de classe
%\item le CO Foron est environ la même longeur qu'une salle de classe
\end{multi}

\begin{multi}[points=1]{Soleil et Terre}
Le diamètre du Soleil est d'environ 1,4 million de km et celui de la Terre est de 13'000 km. Avec ces informations, on peut dire que :
\item le diamètre du Soleil est environ 10 fois plus grand que celui de la Terre
\item * le diamètre du Soleil est environ 100 fois plus grand que celui de la Terre
\item le diamètre du Soleil est environ 1000 fois plus grand que celui de la Terre
\item le diamètre du Soleil est du même ordre de grandeur que celui de la Terre
\end{multi}


\end{quiz}

\begin{quiz}{Puissance de dix}

\begin{multi}[points=1]{}
3,4 milliard en puissance de dix s'écrit ...
\item $3,4\cdot 10^{6}$ 
\item   * $3,4\cdot 10^{9}$ 
\item $3,4\cdot 10^{3}$ 
\item  $34\cdot 10^{6}$ 
\end{multi}

\begin{multi}[points=1]{}
10 millions en puissance de dix s'écrit ...
\item $10^{6}$ 
\item   * $10^{7}$ 
\item $10^{3}$ 
\item  $10^{8}$ 
\end{multi}

\begin{multi}[points=1]{}
un million en puissance de dix s'écrit ...
\item $10^{7}$ 
\item   * $10^{6}$ 
\item $10^{3}$ 
\item  $10^{8}$ 
\end{multi}

\begin{multi}[points=1]{}
$10^5$ = ?
\item 1\,000\,000 
\item   * 100\,000
\item  10\,000
\item  10\,000\,000
\end{multi}

\begin{multi}[points=1]{}
$10^2$ = ?
\item 1\,000
\item   * 100
\item  10
\item  1
\end{multi}

\begin{multi}[points=1]{}
$10^0$ = ?
\item 10
\item   * 1
\item  0,1
\item  0,01
\end{multi}

\begin{multi}[points=1]{}
$10^1$ = ?
\item 1
\item   * 10
\item  0,1
\item  0,01
\end{multi}

\begin{multi}[points=1]{}
$10^{-1}$ = ?
\item 1
\item   * 0,1
\item  10
\item  0,01
\end{multi}

\begin{multi}[points=1]{}
$10^{-2}$ = ?
\item 1
\item   * 0,01
\item  10
\item  0,1
\end{multi}

\begin{multi}[points=1]{}
$10^{-3}$ = ?
\item 1
\item   * 0,001
\item  10
\item  0,1
\end{multi}

\begin{multi}[points=1]{}
$3 \cdot 10^{-3}$ = ?
\item 3
\item   * 0,003
\item  30
\item  0,3
\end{multi}

\begin{multi}[points=1]{}
$7 \cdot 10^{-2}$ = ?
\item 0,02
\item   * 0,07
\item  0,007
\item  0,002
\end{multi}

\end{quiz}

\begin{quiz}{Ecriture scientifique}
\begin{multi}[points=1]{}
3400 en écriture scientifique s'écrit...
\item $34\cdot 10^2$
\item   * $3,4\cdot 10^3$
\item  $3,4\cdot 10^4$
\item  $34\cdot 10^3$
\end{multi}

\begin{multi}[points=1]{}
4800 en écriture scientifique s'écrit...
\item $48\cdot 10^2$
\item   * $4,8\cdot 10^3$
\item  $4,8\cdot 10^4$
\item  $48\cdot 10^3$
\end{multi}

\begin{multi}[points=1]{}
0,3 en écriture scientifique s'écrit...
\item $3\cdot 10^{-2}$
\item   * $3\cdot 10^{-1}$
\item  $30\cdot 10^{-1}$
\item  $30\cdot 10^{0}$
\end{multi}

\begin{multi}[points=1]{}
0,02 en écriture scientifique s'écrit...
\item $2\cdot 10^{-2}$
\item   * $2\cdot 10^{-2}$
\item  $20\cdot 10^{-1}$
\item  $20\cdot 10^{0}$
\end{multi}

\begin{multi}[points=1]{}
0,004 en écriture scientifique s'écrit...
\item $4\cdot 10^{-2}$
\item   * $4\cdot 10^{-3}$
\item  $40\cdot 10^{-1}$
\item  $40\cdot 10^{0}$
\end{multi}

\begin{multi}[points=1]{}
300\,000 en écriture scientifique s'écrit...
\item $3\cdot 10^{4}$
\item   * $3\cdot 10^{5}$
\item  $3\cdot 10^{6}$
\item  $3\cdot 10^{7}$
\end{multi}

\begin{multi}[points=1]{}
$5,2 \cdot 10^{-3}$ en écriture décimale s'écrit...
\item 0,052
\item * 0,005\,2
\item 0,52
\item 5,2
\end{multi}

\begin{multi}[points=1]{}
$1,5 \cdot 10^{8}$ en écriture décimale s'écrit...
\item 1,5
\item * 150\,000\,000
\item 150\,000
\item 15\,000\,000
\end{multi}

\begin{multi}[points=1]{}
$2,5 \cdot 10^{9}$ en écriture décimale s'écrit...
\item 2,5
\item * 2\,500\,000\,000
\item 250\,000,\,000
\item 25\,000\,000\,000
\end{multi}

\end{quiz}
\end{document}
